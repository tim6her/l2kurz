%!TEX root = l2kurz-Tutorium.tex

\section{Literaturangaben}

Mit der \texttt{thebibliography}-Umgebung kann man ein
Literaturverzeichnis erzeugen.
Darin beginnt jede Literaturangabe mit \lstinline|\bibitem|.
Als Parameter wird ein Name vereinbart, unter dem die
Literaturstelle im Text zitiert werden kann, und
dann folgt der Text der Literaturangabe.
Die Nummerierung erfolgt automatisch.
Der Parameter bei \lstinline|\begin{thebibliography}| gibt die
maximale Breite dieser Nummernangabe an, also z.\,B.\
\lstinline|{99}| für maximal zweistellige Nummern.

Im Text zitiert man die Literaturstelle dann mit dem Befehl \lstinline|\cite|
und dem vereinbarten Namen als Argument.

\let\origcite\cite
\begin{LTXexample}[preset=\let\cite\origcite]
Partl~\cite{pa} hat
vorgeschlagen ...

\begin{thebibliography}{99}
\bibitem{pa}
H.~Partl: \textit{German \TeX,}
TUGboat Vol.~9, No.~1 (1988)
\end{thebibliography}
\end{LTXexample}

Werden viele Literatureinträge zitiert bzw. verwendet, bietet sich die Nutzung
einer Datenbank an. Die Datenbank besitzt ihre eigene Syntax,
um die benötigten Literatureinträge zu verwalten. Für die komfortable Verwaltung
von Literaturdatenbanken existieren viele Programme wie beispielsweise JabRef (frei) oder
Endnote (kommerziell). Die Datenbank ist im eigentlichen Sinne eine Textdatei mit
Endung \lstinline|bib|.


Für die Verarbeitung dieser Literaturdatenbanken bieten sich zwei verschiedene
Hilfsmittel für \LaTeX{} an. Die klassische Variante ist \emph{Bib\TeX} in Verbindung
mit einem Literaturverzeichnisstil. Die Anpassung an die eigenen Bedürfnisse
gestaltet sich mehr als schwierig. Daher wurde in den letzten Jahren das \LaTeX{}
Makropaket \emph{biblatex} entwickelt, das alternativ zu \emph{Bib\TeX} das mächtigere
Programm \emph{biber} nutzen kann. Das Makropaket \emph{biblatex} erlaubt die Manipulation
des Literaturverzeichnisses auf \LaTeX"=Ebene. Auf CTAN ist eine deutsche Übersetzung der Dokumenation
verfügbar \cite{biblatex-de}.