%!TEX root = l2kurz-Tutorium.tex

\section{Eigene Befehle}

Um eigene Befehle zu definieren, kann man den Befehl
\begin{lstlisting}
\newcommand{<name>}[<num>][<default>]{<definiton>}
\end{lstlisting}
verwenden.
Diesem muss zumindest der Name des Befehls und ein \LaTeX-Code als Definition übergeben werden.
Eigene Befehle zu definieren, hat zwei Vorteile. Zum einen kann man sich so repetitive Tipparbeit ersparen.

\begin{LTXexample}
\newcommand{\zB}{z.\,B.}

Ich mag Tiere, \zB Vögel.
\end{LTXexample}

Zum anderen kann es helfen, Inhalt und Design zu trennen.
Zum Beispiel könnte in Ihrer Arbeit mehrmals die symmetrische Gruppe der Ordnung $ n! $ vorkommen.
Diese Gruppe wird oft mit $ \mathfrak S_n $ notiert.
Wenn Sie nun einen Befehl der Form
\begin{lstlisting}
\newcommand{\SymG}{\mathfrak S}
\end{lstlisting}
erstellen, dann spart dies nicht nur das ständige Ausschreiben des Befehls;
sie können die Notation für die symmetrische Gruppe nun an einer zentralen Stelle verändern und so dem Wunsch Ihres Betreuers nachkommen, die Gruppe mit $ S_n $ zu bezeichnen.

Besonders praktisch sind eigene Befehle mit Parametern. Schauen wir uns hierzu ein Beispiel an:

Die Notation $ x_1, \ldots, x_n$ für endlich viele Objekte ist in der Mathematik omnipräsent.
Es bietet sich also an, einen eigenen Befehl für diese Notation zu definieren.
\begin{LTXexample}
\newcommand{\seqx}
        {x_1, \ldots, x_n}
$\seqx$
\end{LTXexample}

Dies hat aber einen Nachteil: Was wenn sie auch endlich viele Objekte mit dem Namen $ y_i $ benötigen?
Zu diesem Zweck, kann man einen Befehl mit einem Parameter definieren, dem man den Namen der Objekt übergibt.
\begin{lstlisting}
\newcommand{\seq}[1]{#1_1, \ldots, #1_n}
\end{lstlisting}

Zwischen den eckigen Klammern haben wir nun die Anzahl der Parameter übergeben und mit \verb|#1| können wir in der Definition des Befehls auf den ersten Parameter zugreifen.
Wenn wir nun auch den letzten Index flexibel implementieren wollen, könnten wir unserem Befehl einfach noch einen weiteren Parameter übergeben.
Die Definition des Befehls sieht dann aus wie folgt.
\begin{lstlisting}
\newcommand{\seq}[2]{#1_1, \ldots, #1_#2}
\end{lstlisting}

Beim Aufrufen unseres Befehls müssen nun immer beide Parameter übergeben werden.
Deshalb kann man mit \verb|\newcommand| einen Standardwert für den \emph{ersten} Parameter festlegen.
\begin{LTXexample}
\newcommand{\seq}[2][n]
        {#2_1, \ldots, #2_#1}

$\seq{x}$ und $\seq[m]{y}$
\end{LTXexample}
Beachten Sie, dass die Reihenfolge der Parameter vertauscht wurde.

\LaTeX{} erlaubt es nicht bereits definierte Befehle mit \verb|\newcommand| zu überschreiben.
Es gibt einen speziellen Befehl für das Überschreiben von Befehlsdefinitionen:
\begin{lstlisting}
\renewcommand{<name>}[<num>][<default>]{<definiton>}
\end{lstlisting}
Er verwendet dieselbe Syntax wie \verb|\newcommand|.
