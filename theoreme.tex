%!TEX root = l2kurz-Tutorium.tex

\section{Umgebungen für Theoreme, Definitionen u.\,dgl.}

Im Gegensatz zu anderen Texten folgen mathematische Arbeiten oft einer strengen Gliederung in Definitionen, Theoreme, Lemmata, Beweise, Beispiele, etc.\@ Das Paket \texttt{thmtools} erlaubt das einfache Erstellen von Umgebungen, die diese Gliederung herstellen.
Um das Paket \texttt{thmtools} verwenden zu können ist ein Backend notwendig.
Dazu wird das Paket \texttt{amsthm} geladen (siehe \cite{amsthm}).


Die Definition dieser Umgebungen folgt einem einfachen Schema, dass wir uns im Listing~\ref{thm environments} genauer ansehen.

\begin{figure*}
\begin{example}[caption={Beispiel für die Definition von Theoremumgebungen},
label={thm environments}]
\usepackage{amsthm} % Immer zuerst!
\usepackage{thmtools}

\declaretheorem[
    name=Theorem,
    numberwithin=section
    ]{thm}
	
\declaretheorem[
    name=Lemma,
    numberwithin=section
    ]{lem0}
\end{example}
\end{figure*}

Die erste Definition implementiert eine Umgebung für Theoreme, sie wird mit \cs{begin\{thm\}} geöffnet (Das ist der Ausdruck in den geschwungenen Klammern.).
Weiters wird festgelegt, dass die Umgebung "`Theorem"' heißt und ihre Nummerierung am Anfang jedes Abschnittes zurückgesetzt wird.
Genau so definiert man eine Umgebung für Lemmata (\texttt{lem}). Sehen wir uns ein Beispiel an.

\begin{LTXexample}[firstline=3]
\setcounter{thm}{0}
\setcounter{lem0}{0}
\begin{lem0}
    Sei $n \in \mathbb Z$ 
    beliebig, dann ist $n^2$ 
    genau dann gerade, 
    falls $n$ gerade ist.
\end{lem0}

\begin{thm}
    Die Wurzel aus $2$ ist 
    irrational.
\end{thm}
\end{LTXexample}

Einen kleinen Schönheitsfehler hat diese Konstruktion allerdings: Die Lemmata und Theoreme sollten gemeinsam nummeriert werden. Zu diesem Zweck stellt der Befehl \cs{declaretheorem} die Option \texttt{sibling} zur Verfügung.
Wir definieren die Umgebung für Lemmata mit dieser Option neu:

\begin{example}
\declaretheorem[
    name=Lemma,
    sibling=thm,
    ]{lem}
\end{example}

Im obigen Beispiel erhält man nun den folgenden Output.

\begin{LTXexample}[firstline=2]
\setcounter{thm}{0}
\begin{lem}
    Sei $n \in \mathbb Z$ 
    beliebig, dann ist $n^2$ 
    genau dann gerade, 
    falls $n$ gerade ist.
\end{lem}

\begin{thm}
    Die Wurzel aus $2$ ist 
    irrational.
\end{thm}
\end{LTXexample}

Möchte man ganz auf die Nummerierung verzichten, setzt man die Option \verb|numbered = no|.
Im Paket \texttt{amsthm} ist automatisch eine Umgebung für Beweise namens \texttt{proof} integriert, sodass wir das Lemma einfach beweisen können.
Die Umgebung wird automatisch mit einem Beweisendezeichen ($\square$) beendet.
Weitere Details zu dieser Umgebung finden sich in \cite{amsthm}.

\begin{LTXexample}[firstline=3]
\setcounter{thm}{0}
\setcounter{lem0}{0}
\begin{lem0}
    Sei $n \in \mathbb Z$ 
    beliebig, dann ist $n^2$ 
    genau dann gerade,
    falls $n$ gerade ist.
\end{lem0}

\begin{proof}
    Wenn $n$ gerade ist, 
    so existiert eine 
    weitere ganze Zahl $k$,
    sodass $n = 2 k$ gilt. 
    Durch Quadrieren erhält 
    man $n^2 = 4 k^2$ und 
    $n^2$ ist gerade.
    
    Ist andererseits $n$ 
    ungerade, so gibt es ein 
    $k \in \mathbb Z$, sodass 
    $n = 2 k + 1$ gilt. 
    Es folgt die Identität 
    \[n^2 = 4 k (k + 1) + 1\] 
    und $n^2$ ist ungerade.
\end{proof}
\end{LTXexample}


Der Text innerhalb der Umgebungen, die wir mit \cs{declaretheorem} definiert haben, \emph{kursiv} gesetzt wird. Dies ist für Theoreme und verwandte mathematische Aussagen üblich, nicht jedoch für Definitionen, Bemerkungen oder Beispiele.
Das Paket \texttt{thmtools} stellt zwei vordefinierte Stile -- \texttt{definition} und \texttt{remark} -- zur Verfügung, die den üblichen Stilregeln entsprechen.

\begin{example}
\declaretheorem[
    name=Definition,
    style=definition,
    numbered=no,
    ]{defin}
\end{example}

\begin{LTXexample}
\begin{defin}
    Die Menge der rationalen 
    Zahlen $\mathbb Q$ ist die 
    Menge aller 
    \"Aquivalenzklassen der
    Menge
    \[\lbrace (a, b) \mid 
    a, b \in \mathbb Z, 
    b \neq 0\rbrace\]
    bezüglich der 
    \"Aquivalenzrelation
    \[(a, b) \sim (c, d) 
    :\Leftrightarrow
    ad = cb.
    \]
\end{defin}
\end{LTXexample}

Die Paketdokumentation \cite{thmtools} zu \texttt{thmtools} beschreibt noch viele weitere Anpassungsmöglichkeiten.



