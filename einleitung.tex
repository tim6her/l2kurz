%!TEX root = l2kurz-Tutorium.tex
% Siehe https://github.com/texdoc/l2kurz

\begin{titlepage}
\renewcommand{\thefootnote}{\fnsymbol{footnote}}
{\Huge%
\fontfamily{lmss}\fontseries{sbc}\selectfont
\raggedright
\LaTeX-Tutorium
\rule{\textwidth}{0.75pt}
\par
}
\begin{flushleft}
  \normalsize
  \fontfamily{lmss}\fontseries{sbc}\selectfont
  \today\\[2ex]
  Tim B.~Herbstrith
\end{flushleft}

\vfill

{\parindent=0cm
\LaTeX{} ist ein Satzsystem, das für viele Arten von
Schriftstücken verwendet werden kann, von einfachen Briefen bis zu
kompletten Büchern.  Besonders geeignet ist es für 
wissenschaftliche oder technische Dokumente. \LaTeX{} ist für 
praktisch alle verbreiteten Betriebssysteme verfügbar.
 
Die vorliegende Kurzbeschreibung bezieht sich auf die Version
\LaTeXe\ in der Fassung vom Juni~2001 und sollte für den 
Einstieg in \LaTeX{} ausreichen.  
Eine vollständige Beschreibung enthält das \manual{}
in Verbindung mit der Online-Dokumentation.
}
\setcounter{footnote}{0}
\end{titlepage}


{\parindent=0cm\thispagestyle{empty}

Dieses Dokument basiert auf
\bigskip

\sbLaTeXe-Kurzbeschreibung, Version \lkver, \lkdate, Copyright \copyright{} 1998--2012 M.~Daniel, P.~Gundlach, W.~Schmidt, J.~Knappen, H.~Partl, I.~Hyna veröffentlicht unter der \emph{Open Publication License}, v1.0 oder später (die aktuelle Version ist erhältlich unter
\url{http://www.opencontent.org/openpub/})
\bigskip

und wurde für das \LaTeX-Tutorium von T.~B.~Herbstrith bearbeitet und ergänzt.

Die Bearbeitung erfolgte unter Einhaltung der Lizenzbestimmungen aber \emph{ohne} individuelle Zustimmung zu den Änderungen durch die Autoren und Autorinnen des Originaldokuments.
\bigskip

% Lizenzänderung in Absprache mit Walter Schmidt <-> Patrick Gundlach 19. Juni 2012
{\selectlanguage{USenglish}
This material may be distributed only subject to the terms and
conditions set forth in the \emph{Open Publication License}, v1.0 or
later (the latest version is presently available at
\url{http://www.opencontent.org/openpub/}).}


\bigskip

Die in dieser Publikation erwähnten Software- und Hardware"=Bezeichnungen sind
in den meisten Fällen auch eingetragene Warenzeichen und unterliegen als
solche den gesetzlichen Bestimmungen.

\bigskip

\vfill

Dieses Dokument wurde mit \LaTeX{} gesetzt.
Es ist als Quelltext online erhältlich:
\begin{quote}
\url{https://github.com/tim6her/l2kurz}
\end{quote}

\bigskip

Das Originaldokument ist als Quelltext und im PDF-Format online erhältlich:
\begin{quote}
\url{http://mirror.ctan.org/info/lshort/german/}
\end{quote}
Die Änderungen gegenüber dem Originaldokument sind unter \url{https://github.com/tim6her/l2kurz/blob/master/CHANGES} einzusehen.

}
